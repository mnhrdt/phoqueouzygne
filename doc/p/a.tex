\documentclass[a4paper]{article}      % base article class
\usepackage[utf8]{inputenc}           % allow utf8 input
\usepackage{graphicx}                 % includegraphics

\usepackage{amsmath,amsthm}           % fancier math

\usepackage[osf,sups]{Baskervaldx} % lining figures
%\usepackage[bigdelims,cmintegrals,vvarbb,baskervaldx]{newtxmath} % math font
\usepackage[baskervaldx]{newtxmath} % math font
%\usepackage{cabin}                  % monospace font
%\usepackage[cal=boondoxo]{mathalfa} % mathcal from STIX, unslanted a bit


% global settings
\setlength{\parindent}{0pt}           % no paragraph indentation
\setlength{\parskip}{7pt}             % spacing between paragraphs
\pdfimageresolution 200               % change the default for included png files
\pdfinfoomitdate=1\pdftrailerid{}     % ensure reproducible PDF

% theorem environments
\newtheorem{theorem}{Theorem}
\newtheorem{lemma}[theorem]{Lemma}
\newtheorem{remark}[theorem]{Remark}
\newtheorem{definition}[theorem]{Definition}
\newtheorem{proposition}[theorem]{Proposition}

% enric's macros
\newcommand{\1}{\mathbf{1}}
\newcommand{\Z}{\mathbf{Z}}
\newcommand{\N}{\mathbf{N}}
\newcommand{\R}{\mathbf{R}}
\newcommand{\ud}{\mathrm{d}}
\newcommand{\ds}{\displaystyle}
\newcommand{\abs}[1]{\left|#1\right|}
\newcommand{\paren}[1]{\left(#1\right)}
\newcommand{\pairing}[2]{\left\langle #1,#2\right\rangle}



\begin{document}

{\Large Focusing of Sentinel 1 images}

\vspace{2em}

\section{Context}

Sentinel~1 images are typically used in their SLC form (Single Look Complex),
which are complex-valued rectangular images, whose pixels correspond to
well-determined points on the surface of the Earth.  However, the images they
are also distributed in SAFE form (Standard Archive Format for Europe), which
is the datastream directly acquired from the ground stations.

We are interested in focusing for several reasons
%\begin{itemize}
%	\item 
%\end{itemize}«»

\section{Documentation}



\section{Decoding}

\section{Stripmap focusing}

\section{Topsar focusing}

\section{Other algorithms}

\clearpage
\appendix
\section{The math goes here}

This is a tiny course about radar for people who do not know what an antenna
is.

\paragraph{Basic idea}
A~\emph{linear chirp} is a function of the form~$h(x)=e^{-i\pi x^2}$.  It is
called linear because the instantaneous frequency increases linearly.
The linear chirp is very interesting because it is a square root of
the Dirac delta
\begin{equation}
	h\star h=\delta.
	\label{eq:sqrt}
\end{equation}

Equation~(\ref{eq:sqrt}) is the central idea of radar imaging.  The radar antenna
emits a single chirp~$h(x)$.  Then, reflectors at different distances reflect
the chirp, and the antenna receives back a linear superposition of chirps:
$$
f(x)=\sum_n b_n h(x-a_n).
$$
The function~$f$ is a mess because the supports of the chirps overlap.
However, by correlating~$f$ with the original chirp~$h$ and
using equation~(\ref{eq:sqrt}), we can recover the position and intensity of
each reflector:
$$
f\star h (x) = \sum_n b_n \delta(x - a_n)
$$
This process is called~\emph{focusing} or compression, because it
compresses the support of each chirp into a single impulse.

\paragraph{Computation in the sense of distributions}
Equation~\ref{eq:sqrt} makes no sense because the chirp is not integrable,
and convolution of two non-integrable functions is not a well-defined
function.

However, we can interpret~$h\star h$ in the sense of distributions (see
Equation~(\ref{eq:dirac}) below):
$$
h\star h (y)
=
\int e^{i\pi x^2}e^{-i\pi(x+y)^2}\ud x
=
e^{-i\pi y^2}
\int e^{-2i\pi xy}
\ud x
=
e^{-i\pi y^2}\sqrt{2\pi}\, \widehat{1}\paren{2\pi y}
$$
$$
=
e^{-i\pi y^2}2\pi\delta\paren{2\pi y}
=
e^{-i\pi y^2}\delta(y)
=\delta(y)
$$
thus~$h\star h=\delta$.


\paragraph{Computation on a finite window}
The computation above is somewhat whimsical, but sets the right spirit.  By
multiplying the chirp by a compactly supported window, we can perform all the
computations in the classical sense,  at the price of obtaining an
approximation of a finite impulse, instead of a clean Dirac.

We will consider a windowed linear chirp
$$
h(x)
=
e^{iKx^2}
\,
\1_{[-T,T]}(x)
$$
The parameter~$T$ is the support, and~$K$ is the~\emph{ramp rate}
of the chirp.

\begin{proposition}
	Let~$h$ be a windowed linear chirp as above.  Then
	\[
		h\star h(y)=
		2\paren{T-\abs{y}}
		\,
		\mathrm{sinc}\paren{ Ky\paren{T-\abs{y}} }
		\,
		\1_{[-2T,2T]}(y)
	\]
	(a chirped sinc multiplied by a triangular window).
\end{proposition}

\begin{proof}
	First, we assume that~$y\ge0$ and then we will extend the result to all
	values of~$y$ by symmetry.  We start with the definition of correlation:
	\[
		h\star h (y)
		=
		\int
		h(x)\overline{h(x+y)}\,\ud x
		=
		\int
		e^{iKx^2}
		e^{-iK\paren{x+y}^2}
		\1_{[-T,T]} (x)
		\1_{[-T,T]} (x+y)
		\ud x
	\]
	\[
		=
		e^{-iKy^2}
		\int
		e^{-2iKxy}
		\,
		\1_{[-T,T]} (x)
		\,
		\1_{[-T-y,T-y]} (x)
		\,
		\ud x
	\]
	Now, using the hypothesis that~$y\ge0$, we have
	\[
		\1_{[-T,T]} \1_{[-T-y,T-y]}
		=
%		\1_{[\max(-T,-T-y),\ \min(T,T-y)]}
%		=
		\1_{[-T,\ T-y]}
	\]
	using the convention that~$\1_{[a,b]}=0$ if~$b<a$.
	Thus, for~$y\in[0,2T]$:
	\[
		h\star h (y)
		=
		e^{-iKy^2}
		\int_{-T}^{T-y}
		e^{-2iKxy}
		\ud x
	\]
	and this integral can be computed explicitly and arranged into a convenient
	form:
	\[
		=
		e^{-iKy^2}
		\left[
			\frac{\displaystyle e^{-2iKxy}}{-2iKy}
		\right]_{-T}^{T-y}
	\]
	\[
		=
		\frac{1}{2iKy}
		e^{-iKy^2}
		\paren{
			e^{2iKyT}-e^{-2iKy(T-y)}
		}
	\]
	\[
		=
		2\paren{T-y}
		\frac{
			e^{i\left[2Ky\paren{T-y}\right]}-e^{-i\left[2Ky\paren{T-y}\right]}
			}{
				2i\left[2Ky\paren{T-y}\right]
		}
	\]
	\[
		=
		2\paren{T-y}\mathrm{sinc}\paren{2Ky\paren{T-y}}
	\]
	This gives the result for~$h\star h(y)$ when~$y\in[0,2T]$.  For~$y>2T$ the
	value is~$0$, and by symmetry we can extend the result to~$y<0$ obtaining
	the following expression, valid for any~$y\in\R$:
	\[
		h\star h(y)=\paren{T-\abs{y}}_+
		\,
		\mathrm{sinc}\paren{ Ky\paren{T-\abs{y}} }
	\]
\end{proof}


\paragraph{The two-dimensional case}

\clearpage
\paragraph{Fourier analysis reminder}

We use the unitary convention of Fourier transforms
$$ \hat{u}(y)=\frac{1}{\sqrt{2\pi}}\int u(x)e^{-ixy}\ud x $$
$$ \check{u}(y)=\frac{1}{\sqrt{2\pi}}\int u(x)e^{ixy}\ud x $$
and the following notation for convolution and correlation
$$ f*g(y)=\int f(x)g(y-x)\ud x $$
$$ f\star g(y)=\int f(x)\overline{g(y+x)}\ud x $$
both operations are commutative and covariant to shifts, but only the
convolution is associative.


%The tempered distributions are defined as the dual of the Schwartz space
%$(which is closed by the Fourier transform).
The Fourier transform of a tempered distribution~$T$ is defined by its action
over Schwartz test functions~$\varphi\in\mathcal{S}$ (smooth functions
all of whose derivatives are rapidly decreasing):
$$
\pairing{\hat{T}}\varphi
=
\pairing{T}{\hat{\varphi}}
$$
For example, the Fourier transform of the constant~$1$ is
$$
\pairing{\hat 1}{\varphi}
=
\pairing{1}{\hat \varphi}
=
\int 1\hat\varphi
=
\sqrt{2\pi}\frac{1}{\sqrt{2\pi}}\int 1\hat\varphi
=\sqrt{2\pi}\check{\hat{\varphi}}(0)
=\sqrt{2\pi}\varphi(0)
$$
thus
\begin{equation}
	\hat 1=\sqrt{2\pi}\delta.
	\label{eq:dirac}
\end{equation}

The convolution of two tempered distributions cannot be defined in general
(otherwise you could compute the product of any two distributions via the
convolution theorem).  However, for some particular pairs of
distributions~$S,T$ the following expression makes sense
$$
\pairing{S*T(x)}{\varphi(x)}
=
\pairing{\,S(t)\,}{\,\pairing{T(x)}{\varphi(x+t)}\,}
$$
and in that case, the right hand side can be used as the definition of~$S*T$.



\end{document}

% vim:set tw=77 filetype=tex spell spelllang=en sw=2 ts=2:
