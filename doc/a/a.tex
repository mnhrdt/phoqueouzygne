\documentclass[a4paper]{article}           % base article class
\usepackage[utf8]{inputenc}                % allow utf8 input
\usepackage{graphicx}                      % includegraphics
\usepackage{amsmath,amsthm}                % fancier math
\usepackage{Baskervaldx}                   % beautiful font for text
\usepackage[baskervaldx,upint]{newtxmath}  % beautiful font for math
\usepackage{cabin}                         % beautiful font for code
\usepackage[cache=false]{minted}           % code syntax highlighting

% global settings
\setlength{\parindent}{0pt}                % no paragraph indentation
\setlength{\parskip}{7pt}                  % spacing between paragraphs
\pdfimageresolution 200                    % pixel size for included pngs
\pdfinfoomitdate=1\pdftrailerid{}          % ensure reproducible PDF

% theorem environments
\newtheorem{theorem}{Theorem}
\newtheorem{lemma}[theorem]{Lemma}
\newtheorem{remark}[theorem]{Remark}
\newtheorem{definition}[theorem]{Definition}
\newtheorem{proposition}[theorem]{Proposition}

% enric's macros
\newcommand{\1}{\mathbf{1}}
\newcommand{\Z}{\mathbf{Z}}
\newcommand{\N}{\mathbf{N}}
\newcommand{\R}{\mathbf{R}}
\newcommand{\C}{\mathbf{C}}
\newcommand{\ud}{\mathrm{d}}
\newcommand{\ds}{\displaystyle}
\newcommand{\abs}[1]{\left|#1\right|}
\newcommand{\paren}[1]{\left(#1\right)}
\newcommand{\pairing}[2]{\left\langle #1,#2\right\rangle}


\begin{document}

{\Large Focusing of Sentinel-1 images}
\vspace{2em}

\section{Context}

Sentinel-1 images are typically used in their SLC form (Single Look Complex),
which are complex-valued rectangular images, whose pixels correspond to
well-determined points on the surface of the Earth.  However, the images
are also distributed in SAFE form (Standard Archive Format for Europe), which
is the datastream directly acquired from the ground stations.

We are interested in focusing for several reasons
\begin{itemize}
	\item The SAFE files are about 8 times smaller than the SLC files
	\item Some advanced algorithms cannot work with the processed SLC data
	\item Interferometry of reflectors can be more precise
	\item We want to understand the beauty of the focusing algorithms
\end{itemize}

The SLC is a sampled image or, equivalently, a trigonometric polynomial.  In
some contexts, it may be preferable to use a different data representation
(e.g. a weighted sum of impulses at sub-pixel positions).
Since the SAFE->SLC process is lossy and non-invertible, it may be better to
obtain this representation directly from the SAFE files.

In this report we document the efforts of the CMLA team towards this goal.

\section{Documentation}



\section{Decoding}

\section{Stripmap focusing}

\section{Topsar focusing}

\section{Other algorithms}

\clearpage
\appendix
\section{The math goes here}

This is a tiny course about radar for people who do not know what an antenna
is.

\subsection{Basic idea}
A~\emph{linear chirp} is a function of the form~$h(x)=e^{i\pi x^2}$.  It is
called linear because the instantaneous frequency increases linearly.
The linear chirp is very interesting because it is a ``square root'' of
the Dirac delta
\begin{equation}
	h\star h=\delta.
	\label{eq:sqrt}
\end{equation}

\includegraphics{o/chirp.pdf}
%SCRIPT cat <<END | gnuplot > o/chirp.pdf
%SCRIPT set term pdf
%SCRIPT set zeroaxis
%SCRIPT plot [-4:4] [-2:2] cos(pi*x*x),sin(pi*x*x)
%SCRIPT END

Equation~(\ref{eq:sqrt}) is the central idea of radar imaging.  The radar
antenna emits a single chirp~$h(x)$.  Then, reflectors at different distances
reflect echoes the chirp, and the antenna receives back a linear
superposition of chirps:
$$
f(x)=\sum_n b_n h(x-a_n).
$$
The number~$a_n$ is the distance of the~$n$th reflector, and the
coefficient~$b_n\in[0,1]$ is the amount of signal reflected back to the
antenna.
The function~$f$ is a mess because the supports of the chirps overlap.
However, by correlating~$f$ with the original chirp~$h$ and
using equation~(\ref{eq:sqrt}), we can recover the position and intensity of
each reflector:
$$
f\star h (x) = \sum_n b_n \delta(x - a_n)
$$
This process is called~\emph{focusing} or compression, because it
compresses the support of each chirp into a single impulse.

\subsection{Computation in the sense of distributions}
Equation~\ref{eq:sqrt} makes no sense because the chirp is not integrable,
and convolution of two non-integrable functions is not well-defined
in general.

However, we can interpret~$h\star h$ in the sense of distributions (see
Equation~(\ref{eq:dirac}) below):
$$
h\star h (y)
=
\int e^{-i\pi x^2}e^{i\pi(x+y)^2}\ud x
=
e^{i\pi y^2}
\int e^{2i\pi xy}
\ud x
=
e^{i\pi y^2}\sqrt{2\pi}\, \widehat{1}\paren{-2\pi y}
$$
$$
=
e^{i\pi y^2}2\pi\delta\paren{-2\pi y}
=
e^{i\pi y^2}\delta(y)
=\delta(y)
$$
thus~$h\star h=\delta$.


\subsection{Computation on a finite window}
The computation above is somewhat whimsical, but sets the right spirit.  By
multiplying the chirp by a compactly supported window, we can perform all the
computations in the classical sense,  at the price of obtaining an
approximation of a finite impulse, instead of a clean Dirac.

We will consider a windowed linear chirp
$$
h(x)
=
e^{iKx^2}
\,
\1_{[-T,T]}(x)
$$
The parameter~$T$ is the support, and~$K$ is the~\emph{ramp rate}
of the chirp.

%\includegraphics{o/wchirps.pdf}
%%SCRIPT cat <<END | gnuplot > o/wchirp.pdf
%%SCRIPT set term pdf
%%SCRIPT set zeroaxis
%%SCRIPT sinc(x)=sin(x)/x
%%SCRIPT h(T,K,x)=(abs(x)<T)*cos(K*x*x)
%%SCRIPT s(T,K,x)=2*(abs(x)<2*T)*(T-abs(x))*sinc(K*x*(T-abs(x)))
%%SCRIPT plot [-15:15] [-2:12] (abs(x)<5)*cos(x*x),\
%%SCRIPT  2*(abs(x)<10)*(5-abs(x))*sinc(1*x*(5-abs(x)))
%%SCRIPT END

\includegraphics{o/wchirp.pdf}
%SCRIPT cat <<END | gnuplot > o/wchirp.pdf
%SCRIPT set term pdf
%SCRIPT set zeroaxis
%SCRIPT set title "windowed linear chirp with T=5, K=1"
%SCRIPT plot [-8:8] [-2:2] (abs(x)<5)*cos(1*x*x), (abs(x)<5)*sin(1*x*x)
%SCRIPT END

 %%SCRIPT  (abs(x)<10)*(10-abs(x))*sinc(1*x*(10-abs(x))) lw 3

\clearpage
\begin{proposition}
	Let~$h$ be a windowed linear chirp as above.  Then
	\[
		h\star h(y)=
		\paren{2T-\abs{y}}
		\,
		\mathrm{sinc}\paren{ Ky\paren{2T-\abs{y}} }
		\,
		\1_{[-2T,2T]}(y)
	\]
	(a chirped sinc multiplied by a triangular window).
\end{proposition}

\begin{proof}
	First, we assume that~$y\ge0$ and later we will extend the result to all
	values of~$y$ by symmetry.  We start with the definition of correlation:
	\[
		h\star h (y)
		=
		\int
		\overline{h(x)}h(x+y)\,\ud x
		=
		\int
		e^{-iKx^2}
		e^{iK\paren{x+y}^2}
		\1_{[-T,T]} (x)
		\1_{[-T,T]} (x+y)
		\ud x
	\]
	\[
		=
		e^{iKy^2}
		\int
		e^{2iKxy}
		\,
		\1_{[-T,T]} (x)
		\,
		\1_{[-T-y,T-y]} (x)
		\,
		\ud x
	\]
	Now, using the hypothesis that~$y\ge0$, we have
	\[
		\1_{[-T,T]} \1_{[-T-y,T-y]}
		=
%		\1_{[\max(-T,-T-y),\ \min(T,T-y)]}
%		=
		\1_{[-T,\ T-y]}.
	\]
	%using the convention that~$\1_{[a,b]}=0$ if~$b<a$.
	Thus, for~$y\in[0,2T]$:
	\[
		h\star h (y)
		=
		e^{iKy^2}
		\int_{-T}^{T-y}
		e^{2iKxy}
		\ud x
	\]
	and this integral can be computed explicitly and arranged into a convenient
	form:
	\[
		=
		e^{iKy^2}
		\left[
			\frac{\displaystyle e^{2iKxy}}{2iKy}
		\right]_{-T}^{T-y}
	\]
	\[
		=
		\frac{1}{2iKy}
		e^{iKy^2}
		\paren{
			e^{2iKy(T-y)}
			-
			e^{-2iKyT}
		}
	\]
	\[
		=
		\paren{2T-y}
		\frac{
			e^{i\left[Ky\paren{2T-y}\right]}-e^{-i\left[Ky\paren{T-y}\right]}
			}{
				2i\left[Ky\paren{2T-y}\right]
		}
	\]
	\[
		=
		\paren{2T-y}\mathrm{sinc}\paren{Ky\paren{2T-y}}
	\]
	This gives the result for~$h\star h(y)$ when~$y\in[0,2T]$.  For~$y>2T$ the
	value is~$0$, and by symmetry we can extend the result to~$y<0$ obtaining
	the following expression, valid for any~$y\in\R$:
	\[
		h\star h(y)=\1_{[-2T,2T]}(x)\,\paren{2T-\abs{y}}
		\,
		\mathrm{sinc}\paren{ Ky\paren{2T-\abs{y}} }
	\]
\end{proof}

\begin{proposition}
	Let~$h$ be a windowed linear chirp as above, and let~$s_{K,T}(y)=h\star
	h(y)$.  The following families of functions are approximations of the
	identity (they converge to~$\delta$ in the sense of distributions):
	\begin{enumerate}
		\item For a fixed~$K>0$, the functions~$\varphi_n=s_{K,n}$
		\item For a fixed~$T>0$, the functions~$\varphi_n=s_{n,Kn}$
	\end{enumerate}
\end{proposition}

\subsection{The discrete case}

The computations above model the chirp as a function~$\R\mapsto\C$.  In
practice, the signals are sampled at a discrete set of points.  How close
many points do we need?  Well, notice that the instantaneous frequency at the
end of the interval is~$KT$, thus to sample above the Nyquist frequency we
need a rate higher than~$2KT$, or equivalently more than~$4KT^2$ regular
samples inside the window.  Notice that since the chirp can be discontinuous,
even a good sampling according to this criterion may exhibit some Gibbs
effects near the end of the interval (see the figure for $N=101$ below).
%To avoid this discontinuity, and thus this Gibbs effect, it is better to
%chose~$K$ and~$T$ such that~$KT^2$ is an integer multiple of~$2\pi$.

The following Octave code can be used to test the precision of the impulse
response.  The same code has been used to produce the figures (see the
comments inside the source \TeX\ file).

\begin{minted}{octave}
# code to verify the correctness of the formula for the impulse response
K = 0.6;                              # ramp rate
T = 5;                                # support
N = 1001;                             # number of samples
x = linspace(-3*T, 3*T, N);           # sample positions
d = (x(2)-x(1));                      # sampling step
h = exp(i*K*x.*x) .* (abs(x) < T);    # windowed chirp
f = conv(h, conj(h), "same") * dx;    # discrete correlation (normalized)
w = (abs(x) < 2*T) .* (2*T - abs(x)   # triangular window
s = w .* sinc(K*x.*(2*T-abs(x))/pi);  # impulse response
plot(x,real(f),'+',x,s,'-');          # both curves should look the same
\end{minted}

%SCRIPT cat <<END | octave
%SCRIPT K = 0.6;
%SCRIPT T = 5;
%SCRIPT N = 51;
%SCRIPT x = linspace(-3*T, 3*T, N);
%SCRIPT dx = (x(2)-x(1));
%SCRIPT h = exp(i*K*x.*x) .* (abs(x)<T);
%SCRIPT f = conv(h, conj(h), "same") * dx;
%SCRIPT xf = [x ; real(f) ];
%SCRIPT dlmwrite("o/xf", xf', " ");
%SCRIPT END
%SCRIPT
%SCRIPT cat <<END | gnuplot > o/chirpsamples_51.pdf
%SCRIPT set term pdf
%SCRIPT set zeroaxis
%SCRIPT set title "T=5, K=0.6, N=51, just below Nyquist"
%SCRIPT sinc(x)=sin(x)/x
%SCRIPT K=0.6
%SCRIPT T=5
%SCRIPT plot [-12:12] [-2.5:11] \
%SCRIPT 	"o/xf" using 1:2 w points title "Re(h*h)" lt 2 pt 1, \
%SCRIPT 	(abs(x)<2*T)*(2*T-abs(x))*sinc(K*x*(2*T-abs(x))) w lines title "s" lt 1
%SCRIPT END
\includegraphics{o/chirpsamples_51.pdf}
\includegraphics{o/chirpsamples_101.pdf}
\includegraphics{o/chirpsamples_1001.pdf}
%SCRIPT cat <<END | octave
%SCRIPT K = 0.6;
%SCRIPT T = 5;
%SCRIPT N = 101;
%SCRIPT x = linspace(-3*T, 3*T, N);
%SCRIPT dx = (x(2)-x(1));
%SCRIPT h = exp(i*K*x.*x) .* (abs(x)<T);
%SCRIPT f = conv(h, conj(h), "same") * dx;
%SCRIPT xf = [x ; real(f) ];
%SCRIPT dlmwrite("o/xf", xf', " ");
%SCRIPT END
%SCRIPT
%SCRIPT cat <<END | gnuplot > o/chirpsamples_101.pdf
%SCRIPT set term pdf
%SCRIPT set zeroaxis
%SCRIPT set title "T=5, K=0.6, N=101, a bit above Nyquist"
%SCRIPT sinc(x)=sin(x)/x
%SCRIPT K=0.6
%SCRIPT T=5
%SCRIPT plot [-12:12] [-2.5:11] \
%SCRIPT 	"o/xf" using 1:2 w points title "Re(h*h)" lt 2 pt 1, \
%SCRIPT 	(abs(x)<2*T)*(2*T-abs(x))*sinc(K*x*(2*T-abs(x))) w lines title "s" lt 1
%SCRIPT END

%SCRIPT cat <<END | octave
%SCRIPT K = 0.6;
%SCRIPT T = 5;
%SCRIPT N = 1001;
%SCRIPT x = linspace(-3*T, 3*T, N);
%SCRIPT dx = (x(2)-x(1));
%SCRIPT h = exp(i*K*x.*x) .* (abs(x)<T);
%SCRIPT f = conv(h, conj(h), "same") * dx;
%SCRIPT xf = [x ; real(f) ];
%SCRIPT dlmwrite("o/xf", xf', " ");
%SCRIPT END
%SCRIPT
%SCRIPT cat <<END | gnuplot > o/chirpsamples_1001.pdf
%SCRIPT set term pdf
%SCRIPT set zeroaxis
%SCRIPT set title "T=5, K=0.6, N=1001, way above Niqyist frequency"
%SCRIPT sinc(x)=sin(x)/x
%SCRIPT K=0.6
%SCRIPT T=5
%SCRIPT plot [-12:12] [-2.5:11] \
%SCRIPT 	"o/xf" using 1:2 w points title "Re(h*h)" lt 2 pt 1, \
%SCRIPT 	(abs(x)<2*T)*(2*T-abs(x))*sinc(K*x*(2*T-abs(x))) w lines title "s" lt 1
%SCRIPT END


%\paragraph{The two-dimensional case}

\clearpage
\subsection{Fourier analysis reminder}

We use the unitary convention of Fourier transforms
$$ \hat{u}(y)=\frac{1}{\sqrt{2\pi}}\int u(x)e^{-ixy}\ud x $$
$$ \check{u}(y)=\frac{1}{\sqrt{2\pi}}\int u(x)e^{ixy}\ud x $$
and the following notation for convolution and correlation
$$ f*g(y)=\int f(x)g(y-x)\ud x $$
$$ f\star g(y)=\int \overline{f(x)}g(y+x)\ud x $$
both operations are covariant to shifts, but only the
convolution is commutative and associative.


%The tempered distributions are defined as the dual of the Schwartz space
%$(which is closed by the Fourier transform).
The Fourier transform of a tempered distribution~$T$ is defined by its action
over Schwartz test functions~$\varphi\in\mathcal{S}$ (smooth functions
all of whose derivatives are rapidly decreasing):
$$
\pairing{\hat{T}}\varphi
=
\pairing{T}{\hat{\varphi}}
$$
For example, the Fourier transform of the constant function~$1$ is
$$
\pairing{\hat 1}{\varphi}
=
\pairing{1}{\hat \varphi}
=
\int 1\hat\varphi
=
\sqrt{2\pi}\frac{1}{\sqrt{2\pi}}\int \hat\varphi(x)e^{i0x}\ud x
=\sqrt{2\pi}\check{\hat{\varphi}}(0)
=\sqrt{2\pi}\varphi(0)
$$
thus
\begin{equation}
	\hat 1=\sqrt{2\pi}\delta.
	\label{eq:dirac}
\end{equation}

The convolution of two tempered distributions cannot be defined in general
(otherwise you could compute the product of any two distributions via the
convolution theorem).  However, for some particular pairs of
distributions~$S,T$ the following expression makes sense
$$
\pairing{S*T(x)}{\varphi(x)}
=
\pairing{\,S(t)\,}{\,\pairing{T(x)}{\varphi(x+t)}\,}
$$
and in that case, the right hand side can be used as the definition of~$S*T$.



\end{document}

% vim:set tw=77 filetype=tex spell spelllang=en sw=2 ts=2:
r
